
%% General definitions
\documentclass{article} %% Determines the general format.
\usepackage{a4wide} %% paper size: A4.
\usepackage[utf8]{inputenc} %% This file is written in UTF-8.
\usepackage{tikz}
\usetikzlibrary{arrows.meta}
\usepackage{pgf}
\usepackage[T1]{fontenc} %% Format of the resulting PDF file.
\usepackage{fancyhdr} %% Package to create a header on each page.
\usepackage{lastpage} %% Used for "Page X of Y" in the header.
\usepackage{amssymb}
\usepackage{tikz}  %% Pagacke to create graphics (graphs, automata, etc.)
\usetikzlibrary{arrows}   %% Tikz library for nicer arrow heads
\usepackage[base]{babel} %%language support
\usepackage{hyperref}
\usepackage{lipsum}  %% this is only for template, can be deleted

%% Left side of header
\lhead{\course\\\semester\\Report \titlename}
%% Right side of header
\rhead{\authorname\\Page \thepage\ of \pageref{LastPage}}
%% Height of header
\usepackage[headheight=36pt]{geometry}
%% Page style that uses the header
\pagestyle{fancy}


\begin{document}

%% titlepage for the report
\begin{titlepage}
    \begin{center}
        \vspace*{1cm}

        \Huge
        \textbf{\titlename}

        \vspace{0.5cm}
        \LARGE
        \subtitlename

        \vspace{1.5cm}

        \includegraphics[]{report_picture.png}

        \vspace{1.5cm}

        Project for the course\\
        \course\hspace{0.01em} by\\
        \textbf{\authorname}

        \vfill
        \includegraphics[width=0.3\textwidth]{logo.png}

        \Large
        University \uniname\\
        Switzerland\\
        \semester

    \end{center}
\end{titlepage}

\tableofcontents
\clearpage

\section{Introduction}
Introduce the problem you are trying to solve, motivate why solving this
problem is important, highlight the significance of solving this problem. (What are you trying to
do and why?).\\
\lipsum[1-1] %% showtext that can be deleted

\section{Background}
Introduces relevant background knowledge to understand your work (e.g.,
existing algorithms, protocols, software, and technologies).\\
\lipsum[1-1] %% showtext that can be deleted
\clearpage

\section{Methodology}
Conceptual explanation of implementation, configuration, and setup. (May use
code snippets to explain concepts, but not full code listings.)\\
\lipsum[1-2]
\clearpage

\section{Results}
Show the result/capabilities of your solution through plots, tables, screenshots, etc.
\subsection{first results}
\lipsum[1-2]
\subsection{second results}
\lipsum[1-1]
%% picture
\begin{figure}[h]
    \centering
    %% upload a png to the LaTeX file and change the name
    \includegraphics[]{example.png}
    \caption{This is an example how you include pictures.}
    \label{fig:example}
\end{figure}

\section{Discussion}
Discuss the achieved results (good and bad) and their meaning.\\
\lipsum[1-2]
\clearpage

\section{Conclusion}
Summarize what has been achieved, what is left open from your initial plan, and
how your solution could be further developed through future work.\\
\lipsum[1-2]
\clearpage

\section{Lessons learned}
What did you learn during the project? Which skills did you acquire? What
did you learn about yourself and your role in a team software development project?\\
\lipsum[1-3]
\clearpage

\section{References}
List all references, including third-party software and libraries (check software
release licenses about usage permissions).
\begin{itemize}
    \item \href{https://www.google.com}{Google}
    \item \href{https://stackoverflow.com}{Stackoverflow}
    \item \href{https://chatgpt.com}{ChatGPT}
\end{itemize}
\lipsum[1-1]

\section{Appendix}
Declaration of Independent Authorship and any additional
information (e.g., plots, tables, screenshots) that did not make it into the results section but
are still relevant.\\
\lipsum[1-1]

\end{document}
